% Filename: Conclusion.tex
% Last update: Sat Apr 14 2011 by Brett Burton
%    - created
%
%
%%%%%%%%%%%%%%%%%%%%%%%%%%%%%%%%%%%%%%%%%%%%%%%%%%%%%%%%%%%%%%%%%%%%%%

\section{Conclusion}
\label{sec:Conclusion}

In this paper has outlined a comprehensive pipeline to build an inhomogeneous, anisotropic head and brain model based on human data of all image modalities for use in electroencephalography with an emphasis in forward and inverse problem research as well as visualizations of function MRI data and EEG data. Along with the pipeline, the human data will be released as open-source to enable other scientists to have a starting point to continue further research. Building a model can be time consuming and hinder further important research. This model will allow scientists to have a straight-forward path to building their own model and/or using the model because it is a high-resolution, multi-image data set.

Further investigations from building this pipeline include finding better decimation algorithms for three-dimensional tetrahedral finite element meshes. Due to the number of materials included in the mesh, current decimation algorithms have not been able to further simplify the mesh. More exact sinus and skull segmentation methods would improve the appearance and accuracy of the layers. More exact registration techniques that will provide a better transformation matrix for moving images to DTI space, especially fMRI data. Additions to his data set could include more inclusive methods for functional MRI data into a source localization head model and more specific processing of EEG data for different applications which will make for better visualization. 