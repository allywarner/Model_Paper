% Filename: Conclusion.tex
% Last update: Tues 5/15/2018 by Ally Warner
%    - created
%
%
%%%%%%%%%%%%%%%%%%%%%%%%%%%%%%%%%%%%%%%%%%%%%%%%%%%%%%%%%%%%%%%%%%%%%%

\section{Conclusion}
\label{sec:Conclusion}

In this paper we have outlined a comprehensive pipeline to build an inhomogeneous, anisotropic head and brain model based on human data of all image modalities for use in electroencephalography with an emphasis on forward and inverse problem research as well as visualizations of functional MRI data and EEG data. Along with the pipeline, we have released the human data as open-source to enable other scientists to have a starting point for further research. Building a model can be time consuming and may hinder further important research. This model will allow scientists to have a straightforward path to building their own model and give them access to this model because it is the product of a high-resolution, multi-image dataset. The open-source project will be available in parts for those who want to use only specific aspects of the project. 

Further investigations based on this pipeline include finding more appropriate decimation algorithms for three-dimensional tetrahedral finite element meshes to further reduce the mesh size; more exact sinus and skull segmentation methods to improve the appearance and accuracy of these layers; and more robust registration techniques, which will provide a better transformation matrix for moving images to DTI coordinate space, especially for fMRI data. Additions to this dataset could include more methods to incorporate functional MRI data into source localization simulations and more specific processing of EEG data for different applications, which will result in better visualizations. 