% Filename: Conclusion.tex
% Last update: Sat Apr 14 2011 by Brett Burton
%    - created
%
%
%%%%%%%%%%%%%%%%%%%%%%%%%%%%%%%%%%%%%%%%%%%%%%%%%%%%%%%%%%%%%%%%%%%%%%

\section{Conclusion}
\label{sec:Conclusion}

This paper has outlined a comprehensive pipeline to build an inhomogeneous, anisotropic head and brain model based on human data of all image modalities for use in electroencephalography with an emphasis on forward and inverse problem research as well as visualizations of function MRI data and EEG data. Along with the pipeline, the human data will be released as open-source to enable other scientists to have a starting point for further research. Building a model can be time consuming and hinder further important research. This model will allow scientists to have a straightforward path to building their own model and/or using the model because it is a product of a high-resolution, multi-image dataset.

Further investigations based on this pipeline include finding a more appropriate decimation algorithms for three-dimensional tetrahedral finite element meshes to further reduce the mesh size; more exact sinus and skull segmentation methods to improve the appearance and accuracy of these layers; and more exact registration techniques which will provide a better transformation matrix for moving images to DTI coordinate space, especially fMRI data. Additions to this dataset could include more methods to include functional MRI data into source localization simulations and more specific processing of EEG data for different applications, which will make for better visualization. 