% Filename: intro.tex
% Last update: Monday 11/5/2018 by Ally Warner

%%%%%%%%%%%%%%%%%%%%%%%%%%%%%%%%%%%%%%%%%%%%%%%%%%%%%%%%%%%%%%%%%%%%%%
\section{Introduction}
\label{sec:intro}

Many simulation studies in biomedicine are based on a similar sequence of steps, starting from images, creating geometric models, assigning tissue properties, performing numerical simulations and visualizing the resulting computer model and simulation results, a process known as image-based modeling, simulation, and visualization \cite{SCI:Mac2009a,SCI:Joh2015c,SCI:Joh2012a,SCI:Joh2006a,SCI:Joh2004b}.

We describe a detailed, comprehensive pipeline to build a complete, high-resolution head model for electroencephalographic (EEG) source imaging (ESI) and EEG forward simulation studies.  Steps of the pipeline include image acquisition, preprocessing and registration, image segmentation, finite element mesh generation, simulations, and visualization. This project includes a complete, high-resolution brain segmentation that can be used to create three-dimensional tetrahedral volume and surface meshes. We provide two three-dimensional tetrahedral finite element meshes made from the segmentation of different resolutions that serve as volume conductors to solve forward and inverse EEG problems. We also provide simulation examples of the EEG forward problem, with isotropic and anisotropic systems; functional image data mapped onto a tetrahedral mesh; electroencephalography (EEG) signals mapped onto net electrodes; and diffusion tensor data.

Along with outlining the pipeline, we have made the results of this project available as open-source datasets to aid other scientists in their EEG forward and inverse simulation studies and to help them build new head models more efficiently. The complete head model presented here took approximately one year to complete, in part due to the many options in software and techniques, as well as the complexity of the multimodal imaging data. The segmentation of the image data took approximately 100 hours, mostly dedicated to manual editing. Because the segmentation was high resolution, the resulting tetrahedral meshes became extremely large, which made for very slow simulations. Further simplification methods were developed to achieve moderately sized geometric models.  The head model also includes registration of diffusion tensor data, functional MR data, and high-resolution EEG recordings from 128 and 256 electrodes dense array Philips/EGI sensor nets. The images, data, models, and software are
available at \url{www.sci.utah.edu/SCI_headmodel}.