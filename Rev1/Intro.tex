% Filename: intro.tex
% Last update: Monday 1/30/2019 by Ally Warner

%%%%%%%%%%%%%%%%%%%%%%%%%%%%%%%%%%%%%%%%%%%%%%%%%%%%%%%%%%%%%%%%%%%%%%
\section{Introduction}
\label{sec:intro}

Many simulation studies in biomedicine are based on a similar sequence of steps: image aquisition, creating geometric models, assigning tissue properties, performing numerical simulations, and visualizing the resulting computer model and simulation results. These steps generally describe image-based modeling, simulation, and visualization \cite{SCI:Mac2009a,SCI:Joh2015c,SCI:Joh2012a,SCI:Joh2006a,SCI:Joh2004b}. Image-based modeling is useful for simulating neurological processes and can be used in applications such as electroencephalographic (EEG) forward simulation studies, EEG source imaging (ESI), and brain stimulator simulations. However, no open-source datasets or pipelines currently exist that include both functional magnetic resonance imaging (fMRI) and diffusion tensor imaging (DTI) data, which are both important in generating models of electrical propagation within the brain. 

We developed a comprehensive pipeline to build a complete, high-resolution head model containing both fMRI and DTI data. The model was specifically used for EEG forward simulation studies, but can be subsequently used in ESI applications. We applied this pipeline to a healthy, female subject to develop a dataset for open-source distribution. Currently, this is the only female open-source head-modeling dataset.

In this paper, we describe every step of the pipeline, including image acquisition, preprocessing, registration, and segmentation; finite element mesh generation and simulation; and visualization. We also describe the contents of the open-source dataset, which has been released in conjunction with this paper. The open-source dataset includes raw and simulated data from the subject, intermediate results from each stage of the pipeline, and the software examples used to perform the simulations. This pipeline and dataset will be a valuable addition to the brain-modeling community because technical, resource, and expertise costs have limited the availability of such datasets. 

The images, data, models, and software are available at \url{www.sci.utah.edu/SCI_headmodel}.