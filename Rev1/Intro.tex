% Filename: intro.tex
% Last update: Fri 12/8/17 by Ally Warner

%%%%%%%%%%%%%%%%%%%%%%%%%%%%%%%%%%%%%%%%%%%%%%%%%%%%%%%%%%%%%%%%%%%%%%
\section{Introduction}
\label{sec:intro}

Head models are built and used in many research areas including imaging, mesh generation, and any bioelectric research topic that conducts simulations such as forward and inverse problems \cite{ref:Brette2012}. Imaging is necessary for building any head model, and finding a dataset that includes all imaging modalities is rare since most research projects will only include the modalities necessary to their specific project. In each case, a new head model must be built and this is a time consuming task.

In this project, a detailed comprehensive pipeline to build a complete, high-resolution head model is described. Steps include image acquisition, preprocessing and registration, image segmentation, finite element mesh generation, simulations, and visualization. Included in this paper are two three-dimensional tetrahedral finite element meshes of different resolutions. These serve as volume conductors to solve the forward problem. There are visualizations of the forward problem, with isotropic and anisotropic systems; functional MR data mapped onto a tetrahedral mesh; electroencephalography (EEG) signals mapped onto net electrodes; and of diffusion tensor data. There is a complete, high-resolution brain segmentation that can be used to create three-dimensional meshes. 

Along with outlining the pipeline, this project will be released an open-source dataset. This project took roughly a year to complete due to the many options in software and techniques, as well as the complexity of the data. Segmentation took roughly 100 hours because the imaging was done at the highest resolution available, and needed a lot of manual work. Meshing took several weeks to find an appropriate size without holes. Since the segmentation was complex, and high-resolution, tetrahedral meshes become extremely large which makes for very slow simulations. Further simplification methods failed because of the number of different tissues in the head. Registration of diffusion tensor data and functional MR data was difficult because transformations from other software packages were not compatible with the SCIRun problem solving environment \cite{ref:scirun} and functional MR data has previously not been used in SCIRun. This dataset could provide a starting point for other researchers to begin their projects without the hassle of building an entire model. 