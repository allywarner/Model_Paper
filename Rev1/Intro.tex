% Filename: intro.tex
% Last update: Fri 12/8/17 by Ally Warner

%%%%%%%%%%%%%%%%%%%%%%%%%%%%%%%%%%%%%%%%%%%%%%%%%%%%%%%%%%%%%%%%%%%%%%
\section{Introduction}
\label{sec:intro}

Head models are built and used in many research areas, including imaging, mesh generation, and any bioelectric research topic that requires simulations such as forward and inverse problems \cite{ref:Brette2012}. Imaging is necessary for building any head model, and finding a dataset that includes all imaging modalities is rare since most research projects will include only the modalities necessary to their specific project. In each case, a new head model must be built, which is a time consuming task.

In this project, a detailed comprehensive pipeline to build a complete, high-resolution head model is described. Steps include image acquisition, preprocessing and registration, image segmentation, finite element mesh generation, simulations, and visualization. This paper describes two three-dimensional tetrahedral finite element meshes of different resolutions that serve as volume conductors to solve the forward problem. Included are visualizations of the forward problem, with isotropic and anisotropic systems; functional MR data mapped onto a tetrahedral mesh; electroencephalography (EEG) signals mapped onto net electrodes; and diffusion tensor data. Also included is a complete, high-resolution brain segmentation that can be used to create three-dimensional meshes. 

Along with outlining the pipeline, this project will be released as an open-source dataset. This project took roughly a year to complete due to the many options in software and techniques, as well as the complexity of the data. Segmentation took roughly 100 hours, most being dedicated to manual editing. Meshing took several weeks to find an appropriate size without holes. Since the segmentation was complex, and high-resolution, the tetrahedral meshes became extremely large, which made for very slow simulations. Further simplification methods failed because of the number of different tissues in the head. Registration of diffusion tensor data and functional MR data was difficult because transformations from other software packages were not compatible with the SCIRun problem solving environment \cite{ref:scirun}, and functional MR data has not previously been used in SCIRun.