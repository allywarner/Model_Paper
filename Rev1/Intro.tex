% Filename: intro.tex
% Last update: Sat Apr 1 2015 by Brett Burton
%    - created
%
%%%%%%%%%%%%%%%%%%%%%%%%%%%%%%%%%%%%%%%%%%%%%%%%%%%%%%%%%%%%%%%%%%%%%%
\section{Introduction}
\label{sec:intro}

Introduction text.

%%% Definition of cardiac ischemia
%%%Cardiac ischemia occurs when an imbalance develops within the myocardium in which the demand for oxygen and nutrients exceeds local supply \cite{KKA:Hea94}. Under ischemic conditions, cardiac tissue undergoes structural, metabolic, ionic, and neurochemical changes, which lead to, among other consequences, electrical changes within the heart that can be monitored via electrocardiogram (ECG) measurements. Consequently, ECG markers, in particular shifts in the ST segment, have been used clinically to identify patients with myocardial ischemia; however, variability in the presentation of ischemia, (\emph{i.e.} location, size, and geometry) hinders the ECG, diagnostic accuracy of ischemic heart disease, leading to errors in clinical assessment.  

%%% Clinical motivation for studying ischemia
%%%It is generally accepted that ST segment elevations are associated with fully transmural ischemia and myocardial infarction (MI), whereas ST segment depressions are regarded as markers of subendocardial ischemia \cite{BMB:Hol75}.  Diagnostics based on ST segment shifts, however, yield relatively poor sensitivity and specificity when assessing myocardial ischemia.  As a result, testing strategies in both acute care settings and office visits often result in erroneous disease identification.
%%%For instance, studies have found that hospitals tend to over-admit patients with suspected MI or ischemic disease.  In one study, as many as 72\% of cardiac ICU admissions for MI did not suffer from the condition \cite{BMB:Lee87a}. Aside from acute diagnosis and triage environments, such as hospitals and emergency departments, patients may undergo exercise stress testing to determine if a supply/demand imbalance exists under conditions of elevated stress.  These tests have similarly low sensitivity (68\%) and specificity (77\%) in terms of ECG-based prediction \cite{KKA:Gia89}.  Poor sensitivity and specificity markers suggest that current understanding regarding ischemia, and the electrical consequences associated with it, is incomplete.  

%%% Computer models in ischemia: motivation, typical approaches
%%%To address this gap in understanding, computational forward simulation approaches have been developed to study the progression of ischemia and its resulting electrical outcomes.  The general formulation of such models involves the imposition of a bioelectric source, or sources, within a passive volume conductor.  Electric fields generated by the applied sources extend to the cardiac or body surface, which represents the solution space.  Such models have been useful in elucidating some of the causes of electrical ST segment shifts created by ischemia, but uncertainty associated with the condition still exists, particularly with regard to nontransmural cases.  This uncertainty is due, in part, to simplifications employed in the various modeling techniques, particularly with regard to source representations.  The selection of an appropriate source model is essential to produce useful forward simulations, and many representations have emerged, including single dipole models, current wavefront patterns, transmembrane potential differences, etc.  For purposes of this study, we limit our ischemia model sources to multidimensional, geometric zones that are represented by transmembrane potential differences embedded within the cardiac geometry. 

%[|<Comment: This last sentence doesn't make sense to me.  Is this right: sources that represent the ischemic region and are embedded within the myocardial tissue? If this is more or less right, then how about "sources that represent...and are embedded"?>] 

%%% Current Forward Simulation approaches 
%%%Due to the structure of the myocardium, and specifically the susceptibility of the endocardium to ischemic conditions \cite{KKA:Sal63,KKA:Moi72,KKA:Kje72}, it is  broadly accepted that the ischemic region initiates along the endocardial surface and expands transmurally until reaching the epicardial wall \cite{KKA:Rei77}.  As a result, almost all simulations of nontransmural (or partial thickness) ischemia have maintained the endocardially centered source paradigm, with only few exceptions \cite{BMB:Nie2009,BMB:Swe2010a}.  In most cases, only the shape of the ischemic region varies.  For instance, several groups have chosen to use quadrilateral (2D) or hexahedral (3D) source representations of the ischemic zone, largely due to simplicity in imposing such shapes within their chosen cardiac model \cite{BMB:Hop2005,BMB:Joh2001,BMB:Fra2007,BMB:Jie2010}.  Other groups have used circular or spherical regions \cite{BMB:Mac2007,BMB:Tre2007,BMB:Wil2010}, and still others have applied a more endocardial-conforming ellipsoidal shape \cite{BMB:Li98,BMB:Pul2003}.  

%%% Actual observation of ischemic development
%%%Though parameterized shapes, such as spheres and cubes, are preferred for complex computational modeling, they do not fully reflect the spatial complexity of the ischemic region as it develops.  Experimental observations show that ischemia does not develop as a simplistic shape, centered along the endocardium.  Rather, an ischemic region often originates as multiple, smaller regions that expand over time to form larger, more transmural ischemic zones \cite{KKA:Jen75,KKA:Ste77,BMB:Ara2015}.  Furthermore, it has been observed that ischemic development is more often dispersed throughout the myocardial tissue rather than centered along the endocardium during the first several minutes of ischemia \cite{BMB:Ara2015}.

%%% Study Aim
%% Rough
%%%In this study, we present a forward model that incorporates more realistic ischemic zone geometry and compare those results with those produced by common ischemic zone representations.  Experimental recordings of induced ischemia were used to validate the proposed method before comparisons were made between these results and those produced by reconstructions of the three most used ischemic zone representations  (spheres, hexahedra, and conforming ellipsoids). 
%%%We show that the common source representations of ischemic development inadequately represent the true complexity of the condition. <To be completed>
%single region performance
% single zone ischemic regions were best represented by ... (Do I talk about endo, mid epi studies?
%multi-zone performance
%need for more sophisticated parameterization model

% the multi-zonal ischemic zone geometry more accurately represented resulting potentials along the epicardium.  When potentials were propagated to the torso surface %did the zone matter on the ischemic surface.  IF so, what types of differences were seen?  Fill this in here.


%This paper aims to more accurately predict the extracellular epicardial and body surface potentials, associated with the ischemic condition, by imposing a more realistic, experimentally-derived ischemic zone geometry.  Our observations suggest that the simplified geometric assumptions regarding the shape of the ischemic region, though valuable, do not provide sufficiently accurate solutions to the ischemia based forward problem.  We developed a forward model pipeline in which we imposed our experimentally derived ischemic zone geometry in order to reconstruct epicardial and body surface potentials.  Likewise, we reconstructed the three most common ischemic zone geometries (spheres, hexahedra, and ellipsoids), matching ischemic zone volumes with the original multi-zonal representation.  Through these techniques, we were able to show that a multi-zonal ischemic region more accurately represents the extracellular potentials observed on the cardiac and body surfaces implying that 
